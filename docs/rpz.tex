\documentclass[russian,utf8,emptystyle]{eskdtext}

\newcommand{\No}{\textnumero} % костыль для фикса ошибки

\ESKDdepartment{Федеральное государственное бюджетное образовательное учреждение высшего профессионального образования}
\ESKDcompany{Московский государственный технический университет им. Н. Э. Баумана}
\ESKDclassCode{23 0102}
\ESKDtitle{Курсовая работа по дисциплине}
\ESKDdocName{<<Эксплуатация АСОИиУ>>}
\ESKDauthor{Гуща~А.~В.}
\ESKDtitleApprovedBy{к.т.н. доцент}{Постников В.М.}
\ESKDtitleAgreedBy{к.т.н. доцент}{Постников В.М.}
\ESKDtitleDesignedBy{Студент группы ИУ5-92}{Гуща~А.~В}

\usepackage{multirow}
\usepackage{tabularx}
\usepackage{tabularx,ragged2e}
\usepackage{pdfpages}
\renewcommand\tabularxcolumn[1]{>{\Centering}p{#1}}
\newcommand\abs[1]{\left|#1\right|}

\usepackage{longtable,tabu}

\usepackage{geometry}
\geometry{footskip = 1cm}

\pagenumbering{arabic}
\pagestyle{plain}

\usepackage{setspace}
\usepackage{enumitem}

\usepackage{xcolor}
\usepackage{listings}
\lstset{
    breaklines=true,
    postbreak=\raisebox{0ex}[0ex][0ex]{\ensuremath{\color{red}\hookrightarrow\space}},
    extendedchars=\true,
    basicstyle=\small,
    inputencoding=utf8
}
\lstdefinelanguage{D}{
    keywords = {
    abstract,
    alias,
    align,
    asm,
    assert,
    auto,
    body,
    bool,
    break,
    byte,
    case,
    cast,
    catch,
    cdouble,
    cent,
    cfloat,
    char,
    class,
    const,
    continue,
    creal,
    dchar,
    debug,
    default,
    delegate,
    delete,
    deprecated,
    do,
    double,
    else,
    enum,
    export,
    extern,
    false,
    final,
    finally,
    float,
    for,
    foreach,
    foreach_reverse,
    function,
    goto,
    idouble,
    if,
    ifloat,
    immutable,
    import,
    in,
    inout,
    int,
    interface,
    invariant,
    ireal,
    is,
    lazy,
    long,
    macro,
    mixin,
    module,
    new,
    nothrow,
    null,
    out,
    override,
    package,
    pragma,
    private,
    protected,
    public,
    pure,
    real,
    ref,
    return,
    scope,
    shared,
    short,
    static,
    struct,
    super,
    switch,
    synchronized,
    template,
    this,
    throw,
    true, 
    try,
    typedef,
    typeid,
    typeof,
    ubyte,
    ucent,
    uint,
    ulong,
    union,
    unittest,
    ushort,
    version,
    void,
    volatile,
    wchar,
    while,
    with,
    __FILE__,
    __MODULE__,
    __LINE__,
    __FUNCTION__,
    __PRETTY_FUNCTION__,
    __gshared,
    __traits,
    __vector,
    __parameters,
    }
}

\begin{document}
%\maketitle
%\includepdf[pages={1}]{title.pdf}

\section{Условие задачи}
Имеется k потенциальных точек установки радиовышек. Какое минимальное количество вышек и в каких точках нужно установить, чтобы обеспечить максимальное покрытие сети?

\section{Используемая модель}
Сеть представляется матрицей $n$ на $m$, каждый элемент которой хранит логическое значение: есть связь ИСТИНА, связи нет - ЛОЖЬ. Входными данными модели являются:
\begin{enumerate}[label=\arabic*.]
\item $\{(i,j)\}$ - множество координат потенциальных точек в сети для установки радиовышек длины $k$;
\item $r$ - радиус действия одной вышки в ячейках;
\item $n$ и $m$ - размеры сети в ячейках;
\end{enumerate}

Выходными данными модели являются:
\begin{enumerate}[label=\arabic*.]
\item $\{(k,m)\}$ - множество координат точек, где нужно установить вышки, длины $l$, с условием $l \leq k$.
\end{enumerate}

Входные данные считываются из файла, где в первой строчке записан радиус действия одной вышки в ячейках, во второй строчке размеры сети в ячейках через пробел, а в каждой последующей строчке записаны координаты потенциальных точек в сети для установки радиовышек, координаты разделены пробелом. Координаты отсчитываются с нуля. Пример входного файла:
\begin{verbatim}
2
10 10
0 9
1 5
2 2
3 6
5 1
6 4
8 2
8 7
\end{verbatim}

\end{document}